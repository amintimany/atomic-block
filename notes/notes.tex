\documentclass{article}

\usepackage{geometry}[a4paper]
\usepackage{includes}
\usepackage{iris}
\usepackage{hyperref}

\title{\bfseries Atomic Blocks in \Iris{} for Prophecies}
\author{Amin Timany\\ {imec-DistriNet, KU Leuven}}
\date{March 15, 2019}

\allowdisplaybreaks

\begin{document}
\maketitle

\begin{abstract}
  I explain adding atomic blocks to \Iris{}. I then propose a weakest
  precondition calculus for reasoning inside an atomic block. All of
  these are done in the latest version of \Iris{} as of this writing
  which supports prophecies. I will derive appropriate rules for
  reasoning about programs that use atomic blocks to resolve
  (multiple) prophecies atomically together with other physical
  operations. The idea is that the physical portion of the program
  inside the atomic block behaves atomically, \ie, it is logically
  atomic.
\end{abstract}

\section{Languages in \Iris{}}
In \Iris{} we define a programming language by defining its syntax and
its small-step operational semantics. More precisely, we have to
define the set of expressions, the set of values, evaluation contexts
and a head-step relation on expressions. Let us consider as our
running example the language \TheLang{} a concurrent
$\lambda$-calculus with references and atomic blocks. This language is
fairly close to \emph{heap-lang}, the default language that comes with
the \Coq{} formalization of \Iris{}. The expressions, ranged over by
$\expr$, values, ranged over by $\val$, and evaluation contexts,
ranged over by $\lctx$, are as follows:
\begin{align*}
  \expr \bnfdef{} & n \in \mathbb{N} \ALT \expr \binop \expr \ALT \TT \ALT \True
                    \ALT \False \ALT \If \expr then \expr \Else \expr \ALT
                    \Rec f \var = \expr \ALT \expr~\expr \ALT \\
                  & (\expr, \expr) \ALT \Proj{i} \expr \ALT \Inj{i} \expr \ALT
                    \Match \expr with \var => \expr | \var => {\expr} end \ALT \loc
                    \ALT \Ref(\expr) \ALT \deref \expr \ALT\\
                  & \expr \gets \expr \ALT \CAS(\expr, \expr, \expr) \ALT \Fork{\expr} \ALT
                    \ploc \ALT \createproph \ALT \resolveproph(\expr, \expr) \ALT \AtomicBlock{\expr} \\[1em]
  \val \bnfdef{} & n \in \mathbb{N} \ALT \TT \ALT \True \ALT \False \ALT
                   \Rec f \var = \expr \ALT (\val, \val) \ALT \Inj{i} \val \ALT \loc \ALT \ploc\\[1em]
  \lctx \bnfdef{} & [] \ALT [] \binop \expr \ALT \val \binop [] \ALT
                    \If [] then \expr \Else \expr \ALT []~\expr \ALT \val~[] \ALT
                    ([], \expr) \ALT (\val, []) \ALT \Proj{i} [] \ALT\\
                  & \Inj{i} [] \ALT \Match [] with \var => \expr | \var => {\expr} end
                    \ALT \Ref([]) \ALT \deref [] \ALT [] \gets \expr \ALT \val \gets [] \ALT\\
                  & \CAS([], \expr, \expr) \ALT \CAS(\val, [], \expr) \ALT \CAS(\val, \val, []) \ALT
                    \ALT \resolveproph([], \expr) \ALT \resolveproph(\val, [])\\[1em]
  \binop \bnfdef{} & + \ALT - \ALT = \ALT \le \\
\end{align*}
Here, $\Ref(\expr)$, $\deref \expr$ and $\CAS(\expr, \expr, \expr)$
are the heap operations for allocation, reading from, writing to and
the atomic compare and set operation respectively. The symbol $\loc$
ranges over memory locations. The commands $\createproph$ and
$\resolveproph(\expr, \expr)$ are used respectively to create
prophecies and to resolve prophecies. We write $\ploc$ for prophecy
identifiers. The $\AtomicBlock{\expr}$ is a atomic block. Intuitively
$\AtomicBlock{\expr}$ evaluates to a value $\val$ in \emph{a single
  step of computation} if $\expr$ evaluates to $\val$ in \emph{zero or
  more steps}. We write $\lctx[\expr]$ to denote the expression
resulting from putting $\expr$ in the hole of the evaluation context
$\lctx$.

in \Iris{}, the operational semantics of a language is defined as a
thread-pool operational semantics. In the thread-pool operational
semantics, a thread pool takes a step if one of its threads is an
expression of the form $\lctx[\expr]$ and $\expr$ takes a
head-step. Formally, this happens in two steps. First, based on the
head-step we define a primitive-step (head-step extended to evaluation
contexts). Subsequently, we define a thread-pool operation based on
the primitive step. Therefore, to instantiate \Iris{}'s program logic
with a programming language one needs to simply define the syntax,
\ie, expressions, values and evaluation contexts, and define a
head-step relation between operations. These definitions must of
course satisfy certain axioms for them to interact properly and the
resulting language to make sense. We will discuss these axioms later
on. Before that, let us define the operational semantics of
\TheLang{}. For \TheLang{} we have to extend the primitive step to
also include evaluation of atomic blocks.

For \TheLang{} we define the head step operational semantics as
follows. Here, $\hstep$, $\step$ and $\tpstep$ are the head step,
primitive step and thread-pool step respectively. We annotate head
steps and primitive steps with two lists, \eg,
$\overset{\observation \mid \vec{\expr}}{\hstep}$ and
$\overset{\observation \mid \vec{\expr}}{\step}$. These are respectively
the list of observations made (prophecies resolved) and threads forked
by this step of computation. We use both $\empobs$ for the empty
list. The thread-pool reduction relation is only annotated with
prophecies as it already handles forked threads as shown below. We
write $\state$ for state. The state for \TheLang{} consists of a pair
of a heap (a map with finite support from memory locations to values)
and a finite set of prophecy identifiers consisting of prophecy
identifiers generated so far. We write $\state_h$ and $\state_p$for
the heap and the set of prophecy identifiers that $\state$ consists
of.

The head-step relation is defined as follows:
\begin{mathparpagebreakable}
  \infer{}{(\state, \If \True then \expr_1 \Else \expr_2) \overset{\empobs \mid \empfrk}{\hstep} (\state,\expr_1)}
  \and
  \infer{}{(\state, \If \False then \expr_1 \Else \expr_2) \overset{\empobs \mid \empfrk}{\hstep} (\state, \expr_2)}
  \and
  \infer{}{(\state, (\Rec f \var = \expr)~\val) \overset{\empobs \mid \empfrk}{\hstep} (\state, \subst{\expr}{\var, f}{\val, \Rec f \var = \expr})}
  \and
  \infer{}{(\state, \Proj{i} (\val_1, \val_2)) \overset{\empobs \mid \empfrk}{\hstep} (\state, \val_i)}
  \and
  \infer{}{(\state, \Match \Inj{i} \val with \var => \expr_1 | \var => \expr_2 end) \overset{\empobs \mid \empfrk}{\hstep} (\state, \subst{\expr_i}{\var}{\val})}
  \and
  \infer{\loc \not\in \dom(\state_h)}{(\state, \Ref(\val)) \overset{\empobs \mid \empfrk}{\hstep} ((\state_h \uplus \set{(\loc, \val)}, \state_p), \subst{\expr_i}{\var}{\val})}
  \and
  \infer{(\loc, \val) \in \state_h}{(\state, \deref \loc) \overset{\empobs \mid \empfrk}{\hstep} (\state, \val)}
  \and
  \infer{\state_h = \set{(\loc, \val)} \uplus h}{(\state, \loc \gets \valB) \overset{\empobs \mid \empfrk}{\hstep} ((\set{(\loc, \valB)} \uplus h, \state_p), \TT)}
  \and
  \infer{\state_h = \set{(\loc, \val)} \uplus h}{(\state, \CAS(\loc, \val, \valB)) \overset{\empobs \mid \empfrk}{\hstep} ((\set{(\loc, \valB)} \uplus h, \state_p), \True)}
  \and
  \infer{(\loc, \valB) \in \state_h}{(\state, \CAS(\loc, \val, \valC)) \overset{\empobs \mid \empfrk}{\hstep} (\state, \False)}
  \and
  \infer{}{(\state, \Fork{\expr}) \overset{\empobs \mid \expr}{\hstep} (\state, \TT)}
  \and
  \infer{\ploc \not\in \state_p}{(\state, \createproph) \overset{\empobs \mid \empfrk}{\hstep} ((\state_h, \state_p \uplus \set{\ploc}), \ploc)}
  \and
  \infer{\ploc \in \state_p}{(\state, \resolveproph(\ploc, \val)) \overset{(\loc, \val)\mid\empfrk}{\hstep} (\state, \TT)}
\end{mathparpagebreakable}

In what follows we will define the primitive step relation for
\TheLang{}. In this case, as opposed to ordinary languages in \Iris{}
we have to consider atomic blocks. An atomic block reduces in single
primitive step if the body reduces to a value in zero or more
steps. We define this formally as a pair mutually inductive relations
$\step$ and $\Astep$. The atomic-block relation, $\Astep$, is only
annotated with observations as we do not allow the body of atomic
blocks to fork threads!
\begin{mathparpagebreakable}
  \infer{(\state, \expr) \overset{\observation \mid \vec{\expr}}{\hstep} (\state,\expr')}{(\state, \lctx[\expr]) \overset{\observation \mid \vec{\expr}}{\step} (\state,\lctx[\expr'])}
  \and
  \infer{(\state, \expr) \overset{\observation}{\Astep} (\state,\val)}{(\state, \lctx[\AtomicBlock{\expr}]) \overset{\observation \mid \vec{\expr}}{\step} (\state,\lctx[\val])}
  \and
  \infer{}{(\state, \val) \overset{\empobs}{\Astep} (\state,\val)}
  \and
  \infer{(\state, \expr) \overset{\observation \mid \empfrk}{\step} (\state',\expr')
    \and (\state', \expr') \overset{\observation'}{\Astep} (\state'',\val)}{(\state, \expr) \overset{\observation \app \observation'}{\Astep} (\state,\val)}
\end{mathparpagebreakable}
Notice that if an expression $\expr$ atomic-block steps to another
expression $\expr'$, then $\expr'$, by definition of $\Astep$, must be
a value.

The thread-pool reduction relation is exactly as it is in
\Iris{}. There, it is defined generally for any primitive step
relation.
\begin{mathparpagebreakable}
  \infer{(\state, \expr) \overset{\observation \mid \vec{\expr}_f}{\step} (\state,\expr')}
  {(\state, \vec{\expr}_1; \expr; \vec{\expr}_2) \overset{\observation}{\tpstep} (\state,\vec{\expr}_1; \expr'; \vec{\expr}_2; \vec{\expr}_f)}
\end{mathparpagebreakable}

The axioms that the expressions, values, evaluation contexts and
head-step relation need to satisfy are as follows:\footnote{There
  other axioms in the \Iris{} formalization which arise because in
  \Coq{} we consider two different types for values and expressions,
  here we simply assume that the set of values is a subset of the set
  of expressions. Furthermore, there are axioms regarding injectivity
  of the operation of filling an evaluation contexts with an
  expression.  This is due to the fact that in \Coq{} formalization
  the filling function is taken to be an arbitrary function that takes
  an evaluation context and an expression and produces a resulting
  expression.} Our definition above satisfies these axioms and it is
relatively easy to check this (by simple case analyses on expressions,
values, reduction rules, \etc).

\begin{mathparpagebreakable}
\inferH{Val-Head-Stuck}{}{\forall \state, \state', \expr, \observation, \vec{\expr}.\; (\state, \val) \not \overset{\observation \mid \vec{\expr}}{\hstep} (\state',\expr)}
\and
\inferH{Fill-Val}{\lctx[\expr]~\text{ is a value}}{\expr~\text{ is a value}}
\and
\inferH{Head-Step-Ctx-Val}{(\state, \lctx[\expr]) \overset{\observation \mid \vec{\expr}}{\hstep} (\state',\expr')}{\expr~\text{ is a value}}
\and
\inferH{Atomic-Block-Head-Stuck}{\expr~\text{ is an atomic block, \ie, of the form }~\AtomicBlock{\expr'}}{\forall \state, \state', \expr'', \observation, \vec{\expr}.\; (\state, \expr) \not \overset{\observation \mid \vec{\expr}}{\hstep} (\state',\expr'')}
\and
\inferH{Atomic-Block-Not-Val}{\expr~\text{ is an atomic block, \ie, of the form }~\AtomicBlock{\expr'}}{\expr~\text{ is \emph{not} a value}}
\and
\inferH{Atomic-Block-Not-Ctx}{\expr~\text{ is an atomic block, \ie, of the form }~\AtomicBlock{\expr'}}{\forall \lctx, \expr''.\; \expr \neq \lctx[\expr'']}
\end{mathparpagebreakable}
The axioms \ruleref{Atomic-Block-Head-Stuck},
\ruleref{Atomic-Block-Not-Val}, \ruleref{Atomic-Block-Not-Ctx} new and
are not part of standard \Iris{} as of this writing.\footnote{These
  axioms are for the language construction in our development named
  ``\texttt{atomic\_ectxi\_langage}'' which is a counterpart to the
  ``\texttt{ectxi\_langage}'' from the \Iris{} development. The axioms
  of ``\texttt{atomic\_ectx\_langage}'' which correspond to \Iris{}'s
  ``\texttt{ectx\_langage}'' are slightly more involved. Note that this
  is also the case for the axioms of ``\texttt{ectx\_langage}''
  compared to the axioms of ``\texttt{ectxi\_langage}'' in
  \Iris{}. This is due to the fact that the constructions with more
  slightly more involved axioms make fewer assumptions about the
  structure of the evaluation contexts.}

\section{The Program Logic of \Iris{}: Weakest Preconditions}

\section{Atomic-Block Weakest Preconditions}

\section{Example: Prophecy Resolution Atomically with $\CAS$}

\end{document}
